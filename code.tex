\chapter{Code e FIFO}
\section{Libreria che gestisce le code}
\subsection{-code.h}
\begin{codice}

#include<iostream>
struct nodo {
    char chiave;
    nodo *next;
    nodo(char c='\0', nodo* n=NULL);
};
struct coda {
    nodo *inizio;
    nodo *fine;
    coda(nodo* i=NULL, nodo* f=NULL);
};

void push(char c, coda &Q);
char pop(coda &Q);
bool e_vuota(coda Q);
\end{codice}

	\subsection{-code.cpp}
\begin{codice}

#include<iostream>
#include "code.h"
nodo::nodo(char c, nodo* n) {chiave=c; next=n;}
coda::coda(nodo* i, nodo* f) {inizio=i; fine=f;}
void push(char c, coda &Q) { //push:aggiunge un nuovo elemento alla fine 
// della coda
    if(!Q.inizio) {
        Q.inizio=new nodo(c, 0);
        Q.fine=Q.inizio;
    }
    else {
        Q.fine->next=new nodo(c,0);
        Q.fine=Q.fine->next;
    }
}
void pushList(nodo*L,coda & Q) { //pushList: inserisce una nuova lista alla 
// fine della coda
    if(L) {
        nodo* t=L->next;
        push(L,Q);
        pusLlist(t, Q);
    }
}

char pop(coda &Q) { // pop: elimina il primo elemento della coda e lo ritorna
    char key=(Q.inizio)->chiave;
    nodo* del=Q.inizio;
    Q.inizio=(Q.inizio)->next;
    if(!Q.inizio) Q.fine=NULL;
    delete del;
    return key;
}
bool eVuota(coda Q) { // eVuota : ritorna true sse la coda e' vuota
    return(!Q.inizio);
}
\end{codice}

\section{Strutture di FIFO e nodi FIFO}

\begin{codice}
struct FIFO { //implementazione della struttura coda
    nodo* primo, *ultimo; 
    int dim; 
    FIFO(nodo*a=0,nodo*b=0,int c=0){
        primo=a; ultimo=b; dim=c;}
};

struct nodoFIFO{ //per creare liste di FIFO
    FIFO info; 
    nodoFIFO* next; 
    nodoFIFO(FIFO a=FIFO(), nodoFIFO*b=0){
        info=a; next=b;}
};
\end{codice}

\section{Inserimento di un nodo all'inizio della FIFO}

\begin{codice}
FIFO push_begin(FIFO a, nodo* b){
    if(!a.primo){
        a.primo=a.ultimo=b; 
        b->next=0; return a;
    }
    else{
        b->next=a.primo;
        a.primo=b;
        return a;
        }
}

\end{codice}

\section{Inserimento di un nodo alla fine della FIFO}

\begin{codice}
FIFO push_end(FIFO a, nodo*b){
	b->next=0;
	if(!a.primo)
        a.primo=a.ultimo=b;
    else{
        a.ultimo->next=b; 
        a.ultimo=b;
        }
	return a;
}
\end{codice}

\section{Concatenazione di due FIFO}
\begin{codice}
FIFO concF(FIFO a, FIFO b){
	a.ultimo->next = b.primo;
	a.ultimo = b.ultimo;
	return a;
}
\end{codice}
