\documentclass[12pt,a4paper]{book}

    \usepackage[english,italian]{babel}
    \usepackage[utf8]{inputenc}
    \usepackage[T1]{fontenc}
    
    \usepackage{enumitem} %lists
   
    \usepackage{mathtools} %math package
    \usepackage{amssymb}
    \usepackage{amsmath}
    \usepackage{amsthm}

 \usepackage{listings} %coding environment 

\usepackage{geometry}
\geometry{a4paper,top=2.5cm,bottom=3cm,left=2.5cm,right=2.5cm,bindingoffset=0mm}%margini
\raggedbottom %evita di aggiungere spazi

\usepackage[usenames]{color} %Per permettere la colorazione dei caratteri 
\definecolor{orange}{rgb}{1,0.647,0}
\definecolor{cornflowerblue}{rgb}{0.392,0.584,0.929}
\definecolor{green}{rgb}{0, 0.902, 0.451}
\lstnewenvironment{codice}[1]
{	\lstset
	{	basicstyle=\ttfamily,
		columns=fullflexible,				
		basicstyle=\footnotesize \ttfamily,
  		keywordstyle=\bfseries\color{orange},
 		commentstyle=\color{green},
  		%identifierstyle=\color{cyano},
 		stringstyle=\color{cornflowerblue},
		language=C++,
		%float,
		showstringspaces=false
	}
	\lstset
	{	numbers=left,
		numberstyle=\tiny,
		stepnumber=1,
		numbersep=15pt
	}
}{}

    \usepackage{hyperref}
    \hypersetup{hidelinks}
    