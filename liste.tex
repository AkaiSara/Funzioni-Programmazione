\chapter{Liste e nodi}
\section{Struttura nodo}
\begin{codice}

struct nodo{
	int info; 
	nodo* next; 
	nodo(int a=0, nodo*b=0){info=a; next=b;}
}
\end{codice}

\section{Creazione ricorsiva di una lista}
\subsection{Con sentinella}
\begin{codice}

nodo* leggi() {
    int n;
    cin >> n;
    if(n==-1) return 0;
    else return new nodo(n, leggi());
}
\end{codice}

\subsection{Data la dimensione}
\begin{codice}

nodo* crea(int dim){
   if(dim){
    	int x;
    	cin>>x;
    	return new nodo(x,crea(dim-1));
   }
   return 0;
}
\end{codice}

\section{Ricerca di un valore in una lista (ritorna l’indice (intero) del nodo)}
\begin{codice}

int ricerca(nodo* L, int x) {
    if(L==0) return -1;
    if(L->info==x) return 0;
    if(ricerca(L->next, x)==-1) return -1;
    else return 1+ricerca(L->next, x);
}
\end{codice}

\section{Inserimento di un nodo nel posto giusto (ordine crescente)}
\begin{codice}

nodo* inserisci(nodo* L, int x) {
    if (L==0) return new nodo(x,0);
    if(L->info>x) return new nodo(x, L);
    else return new nodo(L->info, inserisci(L->next,x));
}
\end{codice}

\section{Stampa di una lista}


\subsection{Stampa semplice}
\begin{codice}

void stampa(nodo* L) {
    if(L) {
        cout << L->info << " ";
        stampa(L->next);
    }
}
\end{codice}

\subsection{Stampa elaborata}
\begin{codice}

void stampa_lista(nodo* L) {
    if(L==0) cout << "Lista vuota" << endl;
    else {
        cout <<L->info;
        if(L->next==0) cout << endl;
        else {
            cout << "->";
            stampa_lista(L->next);
        }
    } 
}
\end{codice}


\section{Pattern matching}
\subsection{Funzione che ritorna true se c’è match}
\begin{codice}

bool testa(nodo*& y, int* P, int dimP, nodo*& m){
if(!dimP) {  //caso base: pattern vuoto => match vero
    m=y;
          y=0;
    return true;
}
if(!y) return false; //secondo caso base: lista vuota
if(y->info==*P) return testa(y->next, P+1, dimP-1, m); //primo match: ricorsione
return false; //altrimenti niente match
}
\end{codice}

\subsection{Return match e resto della lista per riferimento}
\begin{codice}

//per riferimento il resto della lista e con il return la lista matchata
nodo* match(nodo* &L, int* P, int dimP) {
if(!L) return 0;
nodo* r=L, *m;  //r e' il puntatore all'inizio (ricorsivo) della lista
if(testa(r, P, dimP, m)) {  //se ha trovato il match...
// (dobbiamo costruire la lista restante)
    L=m; //"salta" il match e lo mette (riferimento) in L
    return r; //restituisce la lista del match
}
else return match(L->next, P, dimP);
}
\end{codice}

\subsection{Per riferimento il matche e il resto col return}
\begin{codice}

//per riferimento la lista matchata e il resto col return
nodo* match(nodo* &L, int* P, int dimP) {
if(!L) return 0;
nodo *r=L, *m;
if(testa(r, P, dimP, m)) {
    L=r;
    return m;
}
else {
    L=L->next;
    r->next=match(L, P, dimP);
    return r;
}
}
\end{codice}

\section{Concatenazione}
\subsection{Concatenazione iterativa}
\begin{codice}
nodo* conc(nodo*L1, nodo*L2){
    if(!L1) return L2;
    if(!L2) return L1;
    nodo* x=L1;
    while(x->next)
        x=x->next;
    x->next=L2;
    return L1;
}
\end{codice}

