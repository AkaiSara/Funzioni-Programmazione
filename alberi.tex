\chapter{Alberi binari}
\section{Costruzione di alberi}
\subsection{Struttura albero}
\begin{codice}

struct nodo{
int info; 
nodo* left,*right; 
nodo(int a=0, nodo* b=0, nodo*c=0){info=a; left=b;right=c;}
};
\end{codice}

\section{Funzione buildtree per costruire l’albero}
\begin{codice}

nodo* buildtree(nodo* r, int n) {
if(!n) return r;
int num;
cin>>num;
r = bilanciato(r,num);
return buildtree(r,n-1);
}
\end{codice}

\section{Funzione bilanciato per bilanciare l’albero(necessario)}
\begin{codice}

nodo* bilanciato(nodo* r, int k) {
if(!r) return new nodo(k,0,0);
if(contanodi(r->left)<=contanodi(r->right)) r->left=bilanciato(r->left, k);
else r->right=bilanciato(r->right, k);
}
\end{codice}

\section{Funzione contanodi per contare il numero di nodi dell’albero}
\begin{codice}

int contanodi(nodo* R) {
if(!R) return 0;
return 1+contanodi(R->left)+contanodi(R->right);
}
\end{codice}

\section{Stampa di un albero}
\subsection{Lineare}
\begin{codice}

void stampa_l(nodo* r) {    
//ricorsivo, stampa sottoalbero sx e sottoalbero dx
if(r){
    cout<<r->info<<'(';
    stampa(r->left);
    cout<<',';
    stampa(r->right);
    cout<<')';
}
else cout << '_';
}
\end{codice}

\subsection{Infissa}
\begin{codice}

void infix(nodo *x) {
if(x) {
infix(x->left);  //stampa albero sinistro
cout<<x->info<<' '; //stampa nodo
infix(x->right);  //stampa albero destro
}
}
\end{codice}

\subsection{Prefissa a salti}
\begin{codice}

int stampa_a_salti(nodo* r, int k, int n) {  //stampa un nodo ogni k attraversati
if(!r) return n;
int salti;
if(n==1) {
    cout << r->info << " ";
    n=k;	
}
else n--;
salti=stampa_a_salti(r->left, k, n);
return stampa_a_salti(r->right, k, salti);
}
\end{codice}

\section{Cercare cammino in un albero}
\begin{codice}

bool cerca_cam(nodo* r, int k, int y, int* C) { 
//ritorna true se c' e' e in tal caso si trova in C
if(r->info==y) k=k-1;
if(k<0) return false;
if(!r->left && !r->right) {
    if(k==0) {
        *C=-1;
        return true;
    }
    return false;
}
if(r->left) {
    if(cerca_cam(r->left, k, y, C+1)) {
        *C=0;
        return true;
    }
}
if(r->right) {
    if(cerca_cam(r->right, k, y, C+1)) {
        *C=1;
        return true;
    }
}
return false;
}
\end{codice}


\section{Restituire una lista ordinata} 
\textbf{che consiste di un numero di nodi pari a quelli di un albero e i cui campi info sono gli stessi dell' albero}
\begin{codice}

nodo* buildList(nodoA* r) {
if(!r) return NULL;
return fuse(new nodo(r->info), fuse(buildList(r->left),buildList(r->right)));
}
\end{codice}

\subsection{Restituire la lista ordinata costruita disponendo in ordine i nodi L1 e L2} \textbf{(necessaria alla funzione sopra)}
\begin{codice}

nodo* fuse(nodo* L1, nodo* L2) {
coda Q=coda();
while(L1||L2) {
    if(L1 && L2) {
        nodo* t;
        if(L1->info<=L2->info) {
            t=L1->next;
            push(L1, Q);
            L1=t;
        } else {
            t=L2->next;
            push(L2,Q);
            L2=t;
        }
    } else if(L1) {
        push_list(L1,Q);
        L1=NULL;
    } else {
        push_list(L2,Q);
        L2=NULL;
    }
}
return Q.inizio;
}
\end{codice}

\subsection{Inserire un valore in un albero in modo che sia ordinato} 
\textbf{(e conseguente buildtree)}
\begin{codice}

nodoA* insert(nodoA*r, char y) {
if(!r) return new nodoA(y);

if(conta_n(r->left)<=conta_n(r->right))
r->left=insert(r->left,y);
else   
r->right=insert(r->right,y);
return r;
}
nodoA* buildtree(nodoA*r, int dim) {
if(dim)
{
char z;
cin>>z;
nodoA*x=insert(r,z);
return buildtree(x,dim-1);
}
return r;
}
\end{codice}

\section{Restituisce la lista concatenata i cui nodi puntano ai nodi dell’albero ordinati secondo l’ordine infisso} 
\textbf{tali che il primo nodo nella lista punti al nodo n-esimo dell’albero secondo l’ordine infisso, e i successivi nodi puntano ad un nodo dell’albero ogni k nodi sempre secondo l’ordine infisso}
\begin{codice}

nodo* B(nodoA* r, int k, int &n) {
if(!r) return 0;
nodo* sx, *dx, *c;
sx=B(r->left, k, n);
if(n==1) {
    c=new nodo(r, NULL);
    n=k;
    sx=fuse(sx, c);
} else n=n-1;
dx=B(r->right, k ,n);
return fuse(sx, dx);
}
\end{codice}

\section{Contare le foglie di un albero}
\begin{codice}

int ContaFoglie(nodo* n)
{
if(!n) return(0);
else {
    if ((n->left==NULL) && (n->right==NULL)) return(1);
    else return ContaFoglie(n->left) + ContaFoglie(n->right);
}
}
\end{codice}

\section{Sapere se un albero è perfettamente bilanciato}
\begin{codice}

bool PerfBil(nodo* n)
{
if (!n) return(true);
else {
if ((n->left==NULL) && (n->right==NULL)) // Se il nodo e' una foglia
  return(true);
else {
  if ((n->left!=NULL) && (n->right!=NULL))
    // Paraticamente: se il nodo ha tutti e due i figli..
    return( PerfBil(n->left) && PerfBil(n->right) );
  else return(false);
}
}
}
\end{codice}

\section{Trovare e restituire un nodo con campo info = y}
\begin{codice}

nodo* trova(nodo* x, char y) {
if(!x) return 0; //se e' vuoto fallisce la ricerca
if(x->info==y) return x; //se e' quello restituisce il nodo
nodo* z=trova(x->left, y); //sottoalbero sx
if(z==0) return z;
return trova(x->right);
}
\end{codice}

\section{Trovare l’altezza di un albero}
\begin{codice}

int altezza(nodo*x) {
if(!x->left && !x->right) return 0;
int a=-1, b=-1;
if(x->left) a=altezza(x->left);
if(x->right) b=altezza(x->right);
if(a>b) return a+1;
else return b+1;
}
\end{codice}

\section{Seguire un percorso dato in un albero}
\begin{codice}

nodo* trova(nodo *x, int *C, int lung) {
if(!x) return 0; //fallito
if(lung==0) return x;  //trovato
if(*C==0) return trova(x->left, C+1, lung-1);  //C=0 vuol dire sx
else return trova(x->right, C+1, lung-1);
}
\end{codice}

\section{Da un albero, ricavare una lista dei nodi con campo info = y ricorsivamente}
\begin{codice}

punt* conc(punt*a,punt*b){
if(!a) return b;
a->next=conc(a->next,b);
return a;
}
punt* buildList(nodo*r,int y){ //albero(r), y e' un intero uguale ad un campo info
if(!r) return 0;
punt*a=buildList(r->left,1);
punt*b=buildList(r->right,1);
if(r->info==y)
    b=new punt(r,b);
return conc(a,b);	
}
\end{codice}